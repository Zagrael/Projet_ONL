\documentclass[12pt, a4paper]{report}
\renewcommand{\thesection}{\arabic{section}}
\renewcommand{\thesubsection}{\arabic{subsection}}
\usepackage[a4paper,left=2cm,right=2cm,top=2cm,bottom=2cm]{geometry}
\usepackage{graphicx}
\usepackage[french]{babel}
\usepackage[fpms]{umons-coverpage}
\usepackage{amsmath}
\usepackage{amssymb}
\usepackage{mathtools}
\umonsAuthor{Sabrine \textsc{Riahi}\\Aymerick \textsc{Soyez}}
\usepackage[utf8]{inputenc}
\umonsTitle{Optimisation Non Linéaire}
\umonsSubtitle {\\ \textbf{Récupération d’une image floutée (deblurring)}}
\umonsSupervisor {Sous la direction de Monsieur Nicolas \textsc{Gillis} et Arnaud \textsc{Vandaele}}
\umonsDate {Décembre 2017}
\umonsDocumentType {Projet d'Optimisation}

\begin{document}
\umonsCoverPage
\tableofcontents
\clearpage
\section{Introduction}
Le problème posé est de déflouter une image dont chaque pixel a été remplacé par une combinaison linéaire des pixels voisins. La matrice de floutage utilisée est donnée. L'objectif de ce projet est donc la résolution du problème suivant :\\
\[\underset{0 \leq x \leq 1}{\mathrm{min}} ||Ax - \tilde{x}||_2^2 + \lambda||x||_2^2\]
où :\\
$A$ est la matrice de floutage\\
$\tilde{x}$ est le vecteur de pixels flouté\\
$\lambda$ est un paramètre positif qui dépend du niveau de bruit\\

\section{Étude de la convexité du problème}
Pour qu'un problème soit convexe, il faut que
\begin{itemize}
\item Son domaine $ D $ soit convexe,
\item $\nabla^2f(x) \geqslant 0, \forall x \in D$
\end{itemize}

Le domaine est décrit par \( D = \left\{ x\ |\ c(x) \geq 0 \right\} \) et est convexe si $c(x)$ est concave.\\
Ici : 


\section{Le problème admet-il un minimum global ?}

\section{Conditions d'optimalité}

\section{Méthode de descente de coordonnées}

\section{Méthode du gradient}

\section{Comparaison des méthodes}

\section{Étude de la sensibilité de la solution}


\end{document}